\cvsection{Motivationsschreiben}

Sehr geehrte Mitglieder der Auswahlkomission, sehr geehrte Frau Kirchmeier,

ich möchte mich mit Vorlage dieser Unterlagen für das Deutschlandstipendium 2018 bewerben.
Ich bedanke mich für die Chance, meine Unterlagen einreichen zu dürfen und bin stolz, in die engere Auswahl für das Stipendium genommen zu werden.

Als ehemaliger Stipendiat ist dies nicht meine erste Bewerbung für das Stipendium.
In den vergangenen Bewerbungsrunden habe ich darüber geschrieben, wo ich mich im Fachbereich und an anderen Stellen engagiere.
Da ich inzwischen in meinem zweiten Mastersemester bin und somit auf das Ende meines Studiums zugehe, möchte ich dieses Jahr zwar auch über Engagement schreiben, aber vor allem darstellen, was ich nach meinem Master plane und wie meine aktuellen Pläne durch die Unterstützung des Deutschlandstipendiums ergänzt würden.

Schon zu Beginn meines Studiums habe ich darüber nachgedacht, nach meinem Master als Doktorand an der Universität zu bleiben.
Im vergangenen Wintersemester habe ich die Chance erhalten, als einer der Ersten in Darmstadt meine Bachelorarbeit am neuen Fachgebiet "Machine Learning" zu schreiben.
Professor Kersting hat sehr aktives Interesse an meiner Arbeit gezeigt und umfangreiches und wertvolles Feedback gegeben.
Durch diese enge Zusammenarbeit mit ihm ist aus meinem Wunsch ein fester Plan geworden: Ich will nach meinem Abschluss promovieren.
Auch wenn die Bachelorarbeit für mich eine große Herausforderung darstellte, vor allem im Bezug auf meine Selbstorganisation, hat mich die tiefe, kontinuierliche Arbeit an einem eigenen, wissenschaftlichen Thema begeistert.
Seit Abgabe der eigentlichen Thesis haben wir, Professor Kersting, ein Doktorand des Fachgebietes und ich, weiter an dem Projekt gearbeitet, mit dem Ziel, bald die Ergebnisse auch auf einer wissenschaftlichen Konferenz oder bei einem Fachjournal zu publizieren.
Durch diese Kooperation habe ich weitere Einblicke in die Arbeit als Wissenschaftler erhalten, die mich auch in meinem Berufswunsch bestärken.

Neben meiner fachlichen Arbeit engagiere ich mich seit 3 Jahren am Fachbereich und auf universitärer Ebene in der studentischen und akademischen Selbstverwaltung.
Dadurch habe ich unschätzbare Eindrücke von der Arbeit "hinter den Kulissen" an einer Universität erhalten und mich an verschiedenen Projekten selbst beteiligt, zum Beispiel bei der Überarbeitung von Studiengängen im Senatsausschuss für Lehre.
Dabei habe ich gemerkt, dass mir neben der fachlichen Arbeit in der Informatik auch eine herausragende und zukunftsfähige universitäre Lehre am Herzen liegt.
Daher will ich in Zukunft eine wissenschaftliche Karriere auch immer mit dem Engagement für die Lehre verbinden, egal ob als selbst Verantwortlicher in Lehrveranstaltungen, oder in Gremiem wie dem Lehr- und Studienausschuss.

Das Deutschlandstipendium bietet mir für diesen Karrierewunsch zum einen natürlich eine finanzielle Absicherung, die mich darin unterstützt, mehr Zeit und Energie in Projekte wie das wissenschaftliche Arbeiten oder mein Engagement am Fachbereich zu investieren, ohne dabei auf weitere Erwerbstätigkeit angewiesen zu sein.
Zum anderen würde es mich auch sehr freuen, wenn ich im Rahmen des Deutschlandstipendiums Kontakte zu einem forschungsstarken Industriepartner knüpfen könnte.
Obwohl ich meine Zukunft aktuell eher in einem akademischen Umfeld sehe, wäre dies sehr wertvoll.
Forschung findet gerade in der Informatik nie in einem Vakuum statt, sondern wird immer auch durch die Industrie gefördert und unterstützt, und findet zu Teilen sogar ganz in Unternehmen statt.
Daher denke ich, dass es auch als Akademiker in spe sehr wichtig ist, ein Netzwerk aufzubauen, welches sowohl andere Wissenschaftler, als auch Industriekontakte umfasst.

Final gibt mir die finanzielle Unterstützung durch das Stipendium auch die Möglichkeit mein ehrenamtliches Engagement außerhalb der akademischen Welt weiter zu führen.
Ich bin seit inzwischen einem Jahr Mitglied des SchLAu Projektes in Hessen.
Dieses Projekt will die Sichtbarkeit sexueller und geschlechtlicher Minderheiten erhöhen, um dadurch für die Rechte und die Anerkennung von Schwulen, Lesben, Bisexuellen, Trans*Menschen, und vielen anderen Minderheiten einzustehen, die oft unter dem Kürzel LGBT* zusammengefasst werden.
Der Zweck des SchLAu Projektes ist es dabei, dass LGBT* Personen selbst in Schulen Workshops halten, um einen Dialog zu ermöglichen.
Dadurch werden Vorbehalte und Berührungsängste abgebaut, da LGBT* Personen keine "Sonderbaren" oder "Anderen" mehr sind, sondern als Menschen mit gleichen Rechten wahrgenommen werden.

Als politisch interessierter und engagierte Student ist es für mich wichtig, nicht still zu bleiben, und den Angriffen, die rechte Parteien und Organisationen auf die Würde von Gruppen und Individuen lancieren, aufrecht entgegen zu treten.
Der Beitrag, den ich im Rahmen des Bildungsprojekts SchLAu dazu leiste, ist nicht laut, oder weit sichtbar, dafür aber individuell und persönlich.
Wir arbeiten mit einzelnen Klassen und kleinen Gruppen in jedem Workshop.
Aber gerade daher halte ich diesen Beitrag für unglaublich wichtig, denn die individuelle Ebene, auf der zwischenmenschliche Kontakte und eine Anerkennung des Gegenübers als Mensch mit gleichen Rechten entsteht, ist für mich eine unabdingbare Basis für gesellschaftlichen Zusammenhalt.
Auch diese Arbeit unterstützen Sie, wenn sie mich als Stipendiaten annehmen.

Ich danke Ihnen für das Lesen meiner Bewerbungsunterlagen und hoffe, dass ich Sie von meinen Ideen und Projekten überzeugen konnte.

Mit freundlichen Grüßen

Claas Völcker

\vspace{1cm}

\paragraph{Anlagen}
\begin{itemize}
    \item Lebenslauf
    \item Bewerbungsformular
    \item Leistungsspiegel
    \item Einschätzung des Studiendekans zum Engagement am Fachbereich (Stand Dezember 2017)
    \item Arbeitszeugnisse Deutsche Bank und SRLabs
    \item Nachweis über den Freiwilligendienst
\end{itemize}
